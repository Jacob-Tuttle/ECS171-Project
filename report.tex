\documentclass{article}
\usepackage{graphicx}
\usepackage{fancyhdr}
\usepackage[margin=1.5in]{geometry}
% Introduction and background (1 page)
% Literature review (1/2 page)
% Dataset Description and exploratory data analysis of the dataset (1-2 page)
% Proposed methodology (1 page)
% Experimental results (2-3 pages)
% Conclusion and discussion -- you can include the project roadmap here too (1/2 page)
% References (no limit)
\linespread{1}
\pagestyle{fancy}
\title{\textbf{Credit Score Classification}}
\author{Kyle Jow, Jimmy Nguyen, Kyle Pickle, Jacob Tuttle, Kris Wong}
\date{December 2023}

\begin{document}
\pagenumbering{gobble}
\maketitle
\newpage
\pagenumbering{arabic}
\pagestyle{fancy}
\fancyhf{} % clear header and footer
\fancyhead[R]{Credit Score Classification}
\fancyfoot[R]{\thepage}
\section*{Introduction and background}
A common problem in the finance and banking industry is assessing the risk 
involved in lending someone money. Typically, a credit score is used to determine 
whether a person is a stable borrower or not. This can be a difficult task with 
many factors to take into account when deciding someone's credit score.
\newpage
\section*{Literature review}

\newpage
\section*{Dataset description and exploratory data analysis}
The dataset contains 27 features. The features we will be focusing on in order
to classify a person's credit score into either good, standard, or poor credit
will be these features: Credit Mix, Number of days past payment due date, Number
of delayed payments, Outstanding debt, Credit utilization, Credit history age.
We make this decision based on how traditional credit scores are determined,
such as a FICO credit score.
\newpage
\section*{Proposed methodology}

\newpage
\section*{Experimental results}

\newpage
\section*{Conclusion and discussion}

\newpage
\section*{References}
\textbf{Dataset}:\\
https://www.kaggle.com/datasets/parisrohan/credit-score-classification/data

\end{document}